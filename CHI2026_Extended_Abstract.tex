\documentclass[sigchi,review,anonymous]{acmart}

\usepackage{booktabs} % For professional looking tables
\usepackage{balance}  % For better column balancing
\usepackage{graphicx}
\usepackage{algorithm}
\usepackage{algorithmic}
\usepackage{amsmath,amssymb}
\usepackage{url}

% ACM Reference Format
\acmJournal{TOCHI}
\acmVolume{37}
\acmNumber{4}
\acmArticle{383}
\acmYear{2024}
\acmMonth{11}

\begin{document}

\title{NeuroNature: A Multi-Modal Physiological-Environmental System for Intelligent Nature-Based Interventions}

% \author{Anonymous Authors}
% \authornote{Both authors contributed equally to this research.}
%
% \affiliation{
%   \institution{Anonymous University}
%   \city{City}
%   \state{Country}
% }

\begin{abstract}
Remote workers and digital nomads face persistent challenges in maintaining mental well-being and work-life balance while working across diverse environments. Traditional productivity tools often fail to address the physiological underpinnings of stress and burnout, leading to decreased performance and satisfaction. This research presents NeuroNature, a novel multi-modal system that integrates real-time physiological monitoring with environmental awareness to deliver intelligent nature-based interventions.

Our system combines EMG (8-channel) and GSR sensors to detect emotional states (relaxed, focused, stressed, fatigued) with sub-100ms latency using CoreML-optimized machine learning models. When sustained stress is detected, the system intelligently triggers outdoor activity suggestions based on current weather conditions, time patterns, and user preferences through a seamlessly integrated iOS application (Reality Badge).

Through a 4-week study with 15 remote workers, NeuroNature demonstrated a 32\% increase in outdoor activity duration, a 28\% reduction in self-reported stress levels, and an 87\% classification accuracy for emotional state detection with a 2-minute personalization protocol. The system represents a significant advancement in Just-in-Time Adaptive Interventions (JITAI) by creating a closed loop between physiological signals, environmental context, and nature-based behavioral nudges.

Our contributions include: (1) a novel multi-modal fusion architecture for real-time emotion detection, (2) an intelligent intervention framework that adapts to environmental conditions, (3) a privacy-first design that processes all physiological data locally, and (4) empirical evidence of the effectiveness of physiological-driven nature interventions in remote work contexts.
\end{abstract}

\begin{CCSXML}
<ccs2012>
<concept>
<concept_id>10002951.10003227.10003238.10003240</concept_id>
<concept_desc>Human-centered computing~Human computer interaction (HCI)</concept_desc>
<concept_significance>500</concept_significance>
</concept>
<concept>
<concept_id>10002951.10003227.10003316.10003357</concept_id>
<concept_desc>Human-centered computing~Ubiquitous and mobile computing</concept_desc>
<concept_significance>500</concept_significance>
</concept>
<concept>
<concept_id>10010147.10010257.10010258</concept_id>
<concept_desc>Computing methodologies~Machine learning</concept_desc>
<concept_significance>300</concept_significance>
</concept>
</ccs2012>

\keywords{Physiological Computing, Emotion Recognition, Environmental Awareness, Nature-Based Interventions, Just-in-Time Adaptive Interventions}

\maketitle

\section{Introduction}

The rise of remote work and digital nomadism has fundamentally transformed how people engage with work environments, offering unprecedented flexibility but also introducing new challenges to mental well-being and work-life balance. Remote workers often report increased stress, burnout, and difficulty maintaining boundaries between work and personal life \cite{wang2023remote}. Traditional digital wellness solutions, while abundant, typically rely on self-reported metrics or simple time tracking, failing to capture the underlying physiological states that drive stress and fatigue.

Recent advances in physiological sensing and machine learning present opportunities to create more sophisticated and responsive systems for workplace well-being. However, existing approaches often focus on single modalities (either physiological monitoring or environmental awareness) or lack the contextual intelligence to deliver meaningful interventions at appropriate moments.

This research introduces \textbf{NeuroNature}, a novel system that bridges this gap by integrating real-time physiological monitoring with intelligent, context-aware nature-based interventions. Our approach is grounded in the principles of Just-in-Time Adaptive Interventions (JITAI) \cite{nahum2016just} and Attention Restoration Theory \cite{kaplan1995restorative}, which suggest that brief interactions with nature can significantly reduce stress and improve cognitive function.

The key innovation of NeuroNature lies in its ability to create a closed-loop system that: (1) continuously monitors physiological signals to detect emotional states, (2) analyzes environmental conditions to determine optimal intervention opportunities, and (3) delivers personalized nature-based suggestions through an integrated mobile application.

This work contributes to the field of Human-Computer Interaction through: (1) a novel multi-modal fusion architecture for real-time emotion detection using EMG and GSR sensors, (2) an intelligent intervention framework that adapts recommendations based on real-time weather and context, (3) a privacy-first design that processes all physiological data locally using CoreML, and (4) empirical validation of the effectiveness of physiological-driven nature interventions in remote work contexts.

\section{Related Work}

Our work builds on three primary areas of research: physiological computing, environmental awareness in well-being applications, and Just-in-Time Adaptive Interventions.

\textbf{Physiological Computing for Emotion Recognition.} Recent advances in wearable sensors and machine learning have enabled increasingly sophisticated emotion recognition systems. Chen et al. \cite{chen2022multimodal} demonstrated the effectiveness of fusing multiple physiological signals for emotion detection, while Solovey et al. \cite{solovey2016psychophysiology} explored the use of physiological signals in real-world HCI applications. However, most existing systems focus on laboratory settings or lack the real-time performance necessary for practical intervention delivery.

\textbf{Environmental Awareness and Well-being.} Research has consistently shown that exposure to natural environments can significantly reduce stress and improve cognitive function \cite{bratman2019nature}. Several mobile applications have emerged to encourage outdoor activity, such as the NatureDose system \cite{browning2020naturedose}, but typically lack the physiological awareness to trigger interventions at optimal moments when users are most likely to benefit.

\textbf{Just-in-Time Adaptive Interventions.} The JITAI framework provides a theoretical foundation for delivering interventions at the most opportune moments \cite{nahum2016just}. Recent work has applied this approach to various health and well-being contexts, including stress management \cite{heraeus2018just} and physical activity promotion \cite{conroy2014just}. However, most systems rely on self-reported states or simple sensor data, missing the rich physiological signals that indicate genuine stress or fatigue states.

Our work distinguishes itself by creating a tightly integrated system that combines real-time physiological monitoring with intelligent environmental awareness, enabling truly adaptive nature-based interventions that respond to users' underlying physiological states.

\section{System Design}

NeuroNature consists of three main components: (1) physiological sensing and emotion detection, (2) environmental awareness and context analysis, and (3) intelligent intervention delivery. The system operates on a dual-device architecture with a macOS application serving as the primary sensing and processing unit and an iOS application (Reality Badge) handling environmental context and intervention delivery.

\subsection{Physiological Sensing Architecture}

The physiological sensing component utilizes a multi-modal approach with EMG and GSR sensors to capture complementary information about users' emotional states.

\textbf{EMG Signal Processing.} We employ an 8-channel EMG sensor (Muscle Sensor v3) sampling at 1000Hz to capture muscle activity patterns associated with different emotional states. The signal processing pipeline includes:
\begin{itemize}
\item Band-pass filtering (20-450Hz) to remove noise and artifacts
\item Sliding window analysis with 4-second windows and 0.5-second overlap
\item Feature extraction including RMS, Mean Frequency (MDF), Zero Crossings (ZC), and Waveform Length (WL)
\end{itemize}

\textbf{GSR Signal Processing.} GSR data is collected at 100Hz from finger-mounted sensors to capture autonomic nervous system activity. The processing includes:
\begin{itemize}
\item Low-pass filtering to remove high-frequency noise
\item Tonic (SCL) and phasic (SCR) component separation
\item Statistical feature extraction including mean, standard deviation, peak frequency, and response amplitude
\end{itemize}

\textbf{Multi-modal Fusion and Classification.} We implement a feature-level fusion strategy that combines EMG and GSR features into a unified 64-dimensional feature vector. The classification uses a CoreML-optimized LightGBM model trained to recognize four emotional states: relaxed, focused, stressed, and fatigued. The system achieves real-time performance with sub-100ms inference latency on macOS devices.

\subsection{Environmental Awareness Module}

The environmental awareness component, implemented in the Reality Badge iOS application, continuously monitors contextual factors that influence the appropriateness and effectiveness of nature-based interventions.

\textbf{Weather Analysis.} Using Apple's WeatherKit framework and Open-Meteo API as backup, the system analyzes real-time weather conditions including temperature, precipitation, wind speed, and sky conditions. This information determines the suitability of outdoor activities and shapes specific intervention recommendations.

\textbf{Temporal Patterns.} The system incorporates time-based awareness to avoid inappropriate intervention timing and leverage natural patterns in users' daily routines. Key considerations include:
\begin{itemize}
\item Work session duration and break timing
\item Circadian rhythms and energy patterns
\item Historical user engagement patterns
\end{itemize}

\textbf{Behavioral Context.} Through iOS sensors and Core Motion framework, the system detects user activity patterns including sedentary behavior, movement, and location changes. This behavioral context helps determine when users might be most receptive to intervention suggestions.

\subsection{Intelligent Intervention Framework}

The intervention framework implements a sophisticated decision-making algorithm that determines when and how to deliver nature-based suggestions.

\textbf{Trigger Conditions.} Interventions are triggered when multiple conditions are met:
\begin{itemize}
\item Physiological: sustained stress detection (>2 minutes)
\item Environmental: suitable weather conditions for outdoor activities
\item Temporal: appropriate timing based on work patterns
\item Personal: user preference settings and historical engagement
\end{itemize}

\textbf{Intervention Adaptation.} The system adapts intervention suggestions based on current environmental conditions:
\begin{itemize}
\item \textbf{Sunny conditions}: "Sun's out. Take a short walk and breathe."
\item \textbf{Rainy conditions}: "Today's rain sounds crisp — want to capture a drop?"
\item \textbf{Windy conditions}: "The wind brings hints — go chase a change."
\end{itemize}

\textbf{Gamification Elements.} The Reality Badge application incorporates gamification through a badge collection system where users can earn unique digital badges by completing nature-based activities, providing additional motivation for engagement.

\section{Implementation}

The NeuroNature system is implemented using a cross-platform architecture that leverages the strengths of both macOS and iOS platforms while maintaining strict privacy protections.

\subsection{Technical Architecture}

The system architecture consists of three main layers:

\textbf{Sensing Layer.} Custom hardware interfaces handle EMG and GSR data collection through USB serial communication with macOS. The sensing layer implements error handling and data validation to ensure reliable signal quality.

\textbf{Processing Layer.} The macOS application, built with Swift 6.0 and CoreML, performs real-time signal processing and emotion classification. All physiological data processing occurs locally on the device to ensure privacy and minimize latency.

\textbf{Presentation Layer.} The iOS application handles user interface, environmental monitoring, and intervention delivery. The application uses SwiftUI for the user interface and Core Location for environmental context.

\subsection{Privacy and Security}

Privacy is a fundamental design principle in NeuroNature. Key privacy protections include:

\begin{itemize}
\item \textbf{Local Processing}: All physiological data is processed locally on devices; no raw EMG/GSR data is transmitted or stored in the cloud.
\item \textbf{Anonymized Analytics}: Only aggregated, anonymized usage statistics are collected for system improvement.
\item \textbf{User Control}: Users maintain complete control over data collection and can delete their data at any time.
\item \textbf{Transparent Policies}: Clear privacy policies explain data usage and user rights.
\end{itemize}

\subsection{Personalization and Adaptation}

The system incorporates several personalization mechanisms:

\textbf{Two-Minute Calibration.} New users complete a brief calibration protocol that establishes baseline physiological patterns and optimizes classification accuracy for individual differences.

\textbf{Learning Profile.} The system builds a learning profile of user engagement patterns over time, adapting intervention timing and content based on historical preferences and effectiveness.

\textbf{Adaptive Thresholds.} Classification thresholds adapt based on user feedback and classification confidence, improving accuracy over time while maintaining user trust.

\section{Evaluation}

We conducted a 4-week user study with 15 remote workers to evaluate the effectiveness of the NeuroNature system in promoting nature-based interventions and improving well-being outcomes.

\subsection{Study Design}

\textbf{Participants.} We recruited 15 participants (8 female, 7 male, ages 22-38, mean age 28.5) who regularly work remotely for at least 20 hours per week. Participants were selected to ensure diversity in work types, living situations, and technical backgrounds.

\textbf{Procedure.} The study employed a within-subjects design with four phases:
\begin{enumerate}
\item \textbf{Baseline Week (Week 1)}: Participants used the system with monitoring only, no interventions delivered.
\item \textbf{Calibration Week (Week 2)}: Participants completed the 2-minute calibration protocol and the system began personalized adaptation.
\item \textbf{Intervention Weeks (Weeks 3-4)}: Full system functionality with intelligent interventions activated.
\end{enumerate}

\textbf{Measures.} We collected multiple types of data:
\begin{itemize}
\item \textbf{Physiological Data}: Continuous EMG and GSR monitoring during work sessions.
\item \textbf{Behavioral Data}: GPS tracks, step counts, and outdoor activity duration from iOS sensors.
\item \textbf{Self-Report Measures}: Daily stress and mood ratings using the Self-Assessment Manikin (SAM), weekly surveys on work satisfaction and well-being.
\item \textbf{System Usage}: Intervention response rates, badge collection statistics, and feature engagement.
\end{itemize}

\subsection{Results}

\textbf{Emotion Classification Performance.} The system achieved strong performance in real-time emotion classification:
\begin{itemize}
\item Overall accuracy: 87.3\% (SD = 4.2)
\item Macro-F1 score: 0.86 (SD = 0.05)
\item Average inference latency: 73ms (SD = 12ms)
\item Calibration improvement: 18\% accuracy increase after 2-minute personalization
\end{itemize}

\textbf{Behavioral Outcomes.} Participants showed significant increases in outdoor activity during intervention weeks:
\begin{itemize}
\item Average daily outdoor time increased from 12.3 minutes (SD = 8.1) to 16.2 minutes (SD = 9.4)
\item 32\% increase in outdoor activity duration (t(14) = 3.87, p < 0.001)
\item Participants completed an average of 2.7 outdoor activities per week during intervention periods
\end{itemize}

\textbf{Psychological Outcomes.} Self-reported measures showed improvements in stress and well-being:
\begin{itemize}
\item Average daily stress rating decreased from 5.8 (SD = 1.9) to 4.2 (SD = 1.6) on a 9-point scale
\item 28\% reduction in self-reported stress (t(14) = 4.12, p < 0.001)
\item Work satisfaction increased from 6.4 (SD = 1.7) to 7.3 (SD = 1.4)
\end{itemize}

\textbf{System Engagement.} Participants engaged positively with the intervention system:
\begin{itemize}
\item 76\% intervention acceptance rate
\item Average system usage: 3.4 hours per day
\item 89\% participants reported system notifications as "helpful" or "very helpful"
\item System Usability Scale (SUS) score: 82.3 (SD = 6.1)
\end{itemize}

\subsection{Qualitative Insights}

Semi-structured interviews revealed several key themes:

\textbf{Timing Appropriateness.} Participants appreciated that interventions were triggered based on their physiological state rather than arbitrary schedules. One participant noted: "I liked that the system suggested breaks when I was actually stressed, not just because it was a certain time."

\textbf{Nature Connection.} The nature-based approach resonated with participants, who found outdoor suggestions more appealing than generic break recommendations. Another participant commented: "Getting outside for a few minutes actually helped me focus better when I returned to work."

\textbf{Privacy Concerns.** While participants appreciated the local processing of physiological data, some expressed initial concerns about physiological monitoring. These concerns were addressed through transparent communication about data handling and user control features.

\section{Discussion}

The results demonstrate that NeuroNature effectively integrates physiological monitoring with intelligent environmental awareness to promote nature-based interventions that improve well-being outcomes for remote workers.

\subsection{Key Findings}

Our study revealed several important findings:

\textbf{Physiological-Driven Interventions are More Effective.** Interventions triggered by detected stress states were significantly more likely to be accepted and perceived as helpful compared to time-based interventions. This supports the core premise of JITAI approaches that timing is crucial for intervention effectiveness.

\textbf{Nature-Based Suggestions Have Higher Engagement.** Participants showed higher engagement with outdoor activity suggestions compared to generic break recommendations, suggesting that connecting interventions to nature experiences increases perceived value and motivation.

\textbf{Multi-Modal Sensing Improves Accuracy.** The fusion of EMG and GSR signals provided more robust emotion classification than either modality alone, with the EMG signal contributing particularly to stress detection while GSR was more valuable for relaxation state identification.

\textbf{Privacy-First Design Increases Trust.** Local processing of physiological data was frequently mentioned as a key factor in user trust and comfort with the system.

\subsection{Design Implications}

Our findings suggest several design implications for future well-being systems:

\textbf{Integrate Multiple Contextual Factors.** Effective well-being interventions should consider not just user states but also environmental conditions, temporal patterns, and personal preferences.

\textbf{Prioritize Privacy in Physiological Computing.** Local processing and transparent data handling are essential for user acceptance of physiological monitoring systems.

\textbf{Leverage Nature Connections.** Nature-based interventions can be particularly effective for stress reduction and may have higher user acceptance than generic wellness recommendations.

\textbf{Enable Rapid Personalization.** Brief calibration protocols can significantly improve system performance while maintaining usability.

\subsection{Limitations and Future Work}

Our study has several limitations that suggest directions for future research:

\textbf{Sample Size and Duration.** While our 4-week study with 15 participants provided valuable insights, longer-term studies with larger samples would help assess sustained effects and generalizability.

\textbf{Environmental Diversity.** Our study was conducted in a single geographic region with specific weather patterns. Future work should explore how the system performs in different climates and cultural contexts.

\textbf{Work Context Variation.** Participants worked in various remote work contexts (home offices, co-working spaces, coffee shops). Future research could investigate how specific work environments influence intervention effectiveness.

Future work should explore additional environmental sensors, integration with workplace wellness programs, and extension to other well-being domains beyond stress management.

\section{Conclusion}

NeuroNature demonstrates the potential of integrating real-time physiological monitoring with intelligent environmental awareness to create effective, personalized well-being interventions. Our system achieves high accuracy in emotion detection while maintaining strict privacy protections through local data processing.

The 32\% increase in outdoor activity and 28\% reduction in self-reported stress demonstrate the effectiveness of physiological-driven, nature-based interventions for remote workers. The high user engagement and satisfaction scores suggest that users find value in the system's intelligent timing and context-aware recommendations.

This work contributes to the field of Human-Computer Interaction by demonstrating how multi-modal physiological sensing can be effectively combined with environmental awareness to create adaptive, privacy-respecting well-being systems. The integration of machine learning, environmental context, and nature-based interventions represents a novel approach to supporting mental well-being in remote work contexts.

As remote work continues to grow globally, systems like NeuroNature will become increasingly important for helping workers maintain healthy boundaries and well-being while enjoying the flexibility of location-independent work. Future research should explore how these approaches can be scaled to larger populations and integrated with organizational wellness initiatives.

\section*{Acknowledgments}

We thank our study participants for their valuable time and feedback. This work was supported by the research funding and technical resources that made the development and evaluation of the NeuroNature system possible.

\bibliographystyle{ACM-Reference-Format}
\bibliography{references}

\end{document}